\section{Conclusions}

In this project, 10 fully developed models that predict bus journey times in London have been presented. The best performing model is a combined regression, interpolation and weighted average model that is able to outperform Transport for London's own predictions using the root mean squared error and mean absolute error as the measure of performance. For stops that are close to each other, the predicted journey time is usually within one minute of the actual journey time. For stops that are further away, this error increases such that on average the model predicts a journey time that is within two and a half minutes of the actual journey time. As a bus user, this is not a bad result at all - two to three minutes does not generally make a difference to travel time. Furthermore, this model fulfills the project objective of actively reacting to delays. \\

Currently, the models developed in this project are only compared against each other and against TfL Countdown's own predictions. None of the models made use of traffic conditions and it has been assumed that Countdown also does not look at factors such as congestion. On the other hand, other applications such as CityMapper or Google Maps provide adjustable predicted journey times based on traffic conditions. So, in the future, it would be interesting to see a direct comparison of this project's models against CityMapper and Google Map's predictions. This should provide insight into whether traffic conditions is necessary for vehicle journey predictions. Furthermore, if the phone app is extended such that predictions from different sources can be displayed side by side, this could be useful for users so that they can have multiple predictions to choose from. \\

It was hypothesised that the journey time of a bus would rely equally on both `local' conditions and `global' conditions. This was not proven correct as the best model implied that `local' conditions are much more important than `global' ones. However, due to limited time and resources, it was not possible to collect data on other `global' factors such as weather or passenger counts at each bus stop. Therefore, it would be interesting to see if including these factors in the part 1 regression model would lead to an $\alpha$ value that was nearer to 0.5. In other words, whether including more `global' factors would allow this hypothesis to be true. \\

Artificial Neural Networks were a type of model that was not prioritised for this project, although good results had been found by other papers that implemented this method. However, given more time, it would be interesting to see how an ANN model would compare to the historical average and regression + interpolation combined model. \\

Finally, since the data used for the modelling was all collected manually, there is a high possibility of human error in the code written to do the collecting. Therefore, it is believed that with official historical data, these models could perform even better.


\clearpage