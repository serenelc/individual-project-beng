\section{Introduction}

\subsection{Motivation}

For the year ending in June 2019, the number of local bus passenger journeys in London decreased by 0.9\% \cite{quarterly-bus-stats}. This was the 4th consecutive year in which passenger journeys on buses in London have fallen. Transport for London (TfL) attributes this decrease in bus popularity to a downward trend in bus performance, for example, slower bus speeds or unreliable bus arrival times \cite{annual-bus-stats}. The latter issue is what this project will focus on. \\ 

It is important that more people use buses as a means of transportation and reduce their use of private vehicles. London's population is set to expand from 8.7 million in 2017 to 10.5 million by 2041, generating more than 5 million additional trips each day across the transport network \cite{mayor-reduce-car-use}. Traffic congestion is increasing and air quality is worsening. Although Sadiq Khan, who was elected the London Mayor in 2016, has taken steps to try to encourage more use of public transportation, for example by introducing Ultra Low Emission Zone (ULEZ) charges in 2019, as of February 2020 the car is still the most popular single mode of travel for Londoners, with 29.8 \% of workers opting to commute in this manner \cite{motoring-faqs}. More than 50\% of London's toxic air pollution is caused by vehicles and more than 2 million Londoners live in areas that exceed legal limits for NO2 \cite{mayor-car-free-day}. Therefore, encouraging bus use through improving the experience of using a bus, will help to improve the air quality and the environment overall.

\subsection{Aims}

This project aims to investigate various journey time prediction models and implement the best performing model in a simple mobile app. The mobile app serves merely as a way of displaying the model's results. This project also aims to study various parameters that could affect the journey time of a bus, ranging from time of day to pre/post COVID-19 lockdown. \\

Although there has been a fair amount of research that has gone into the topic of bus journey time modelling, many of them are based on static models \cite{ann-prediction}, \cite{weather-transport-effect}, \cite{google-machine-learning}. This project will focus on dynamic models because it is hypothesised that recent information on bus journey times has more of an impact on predictions than merely using static data. Of course dynamic modelling has been researched \cite{traffic-modelling-article}, \cite{dynamic-gps}, \cite{smart-public-transport}. However, these papers tend to use GPS co-ordinates to aid in their predictions, which is not information that is explored and used in this project.

\subsection{Overview}

The models that will be developed to predict the arrival times of the buses will make use of TfL's publicly provided APIs \cite{tfl-api}. Currently TfL provides a variety of APIs including ones that describe bus routes or bus stop locations. In fact, TfL also provides an arrival prediction API \textit{Countdown}, for both bus routes and tube lines. However, it can be inferred that this API is static in its implementation, meaning it does not use new/current information to aid in its prediction. For a particular bus, say 452, and a particular stop, say `York House Place', Countdown makes use of the timetabled arrival time and also looks at the GPS position of 452 relative to `York House Place', in order to make a prediction for its arrival time there. This project will be exploring both static and dynamic models, but Countdown will be used as the baseline model for which to compare the models developed in this project against. \\

Historical average models, regression models, interpolation models, kalman filter models and artificial neural network models, are explored however, only the first three models are implemented and investigated further. Historical average models were chosen because they are generally suitable for real time predictions, due to the more simplistic nature of the algorithms and smaller computation time \cite{dynamic-gps}. Regression and interpolation models were chosen because they are able to work satisfactorily if traffic conditions are unstable. It was initially planned to implement artificial neural network models too since many papers demonstrated that artificial neural network models outperform historical and regression models \cite{ann-prediction}. However, this was never implemented due to lack of time. Furthermore, research has shown that artificial neural networks have the potential to overfit and so results obtained might not generalise well to routes not used in training \cite{dynamic-gps}. Kalman models were not considered to be implemented because of time constraints and because the mathematical background required was unfamiliar. Also, Kalman models are not the popular choice of model in other papers that have researched bus journey predictions. \\

It is hypothesised that a number of factors including traffic conditions, weather conditions and time of day, can cause a bus to deviate from its scheduled journey time. The factors that have occurred most often in the research papers are explored further in order to decide which ones are more important. The time of day and day of week are implemented in the models developed as they were found to be statistically significant \cite{apc-estimation}. \\

If this study is successful, the end product should be a model that predicts the arrival time of any bus on any route in London more accurately than the Countdown API. Bus passengers want reliable and timely services, but if that is not possible, they want to know how long it will take them to reach their destination. Passengers want to be able to take the bus if their journey time would be shorter than normal or avoid taking the bus if the journey will be significantly longer than usual. By providing a service that supplies users with reliable bus journey times, it is intended that more people will consider the bus over the car as their primary mode of transport. \\

However, it is entirely plausible that the dynamic models could prove less successful than the static model used by Countdown. If that is the case, a thorough evaluation will be done to investigate why. The success of the model is measured by the mean absolute error and root mean squared error of its predictions compared to the actual arrival times and compared to Transport for London's own predictions. 

\clearpage

