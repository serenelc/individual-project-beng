\section{Introduction}

\subsection{Motivation}

For the year ending in June 2019, the number of local bus passenger journeys in London decreased by 0.9\% \cite{quarterly-bus-stats}. This was the 4th consecutive year in which passenger journeys on buses in London have fallen. Transport for London (TfL) attributes this decrease in bus popularity to a downward trend in bus performance, for example, slower bus speeds or unreliable bus arrival times \cite{annual-bus-stats}. The latter issue is what this project will focus on. \\ 

This project aims to BLAH BLAH BLAH  This project aims to study various parameters that could affect the journey time of a bus, ranging from time of day to weather conditions to congestion, and implement a number of models based on these factors to predict bus arrival times. The best performing model will then be used to create a simple mobile or web application that allows users to see the dynamically predicted arrival time from any bus stop for the bus that they intend to take \\

It is important that more people use buses as a means of transportation and reduce their use of private vehicles. London's population is set to expand from 8.7 million in 2017 to 10.5 million by 2041, generating more than 5 million additional trips each day across the transport network \cite{mayor-reduce-car-use}. Traffic congestion is increasing and air quality is worsening. Although Sadiq Khan, who was elected the London Mayor in 2016, has taken steps to try to encourage more use of public transportation, for example by introducing Ultra Low Emission Zone (ULEZ) charges in 2019, as of February 2020 the car is still the most popular single mode of travel for Londoners, with 29.8 \% of workers opting to commute in this manner \cite{motoring-faqs}. More than 50\% of London's toxic air pollution is caused by vehicles and more than 2 million Londoners live in areas that exceed legal limits for NO2 \cite{mayor-car-free-day}. Therefore, encouraging bus use through improving the experience of using a bus, will help to improve the air quality and the environment overall.

\subsection{Contributions}

The models that will be developed to predict the arrival times of the buses will make use of TfL's publicly provided APIs \cite{tfl-api}. Currently TfL provides a variety of APIs including ones that describe bus routes or bus stop locations, as well as ones that provide the live times for when a bus arrives at a specific stop. In fact, TfL also provides an arrival prediction API \textit{Countdown}, for both bus routes and tube lines. This perhaps could be seen as negating the point of the project as described previously. However, it can be inferred that this API is static in its implementation, meaning it does not use new/current information to aid in its prediction. For a particular bus, say 452, and a particular stop, say `York House Place', Countdown makes use of the timetabled arrival time and also looks at the GPS position of 452 relative to `York House Place', in order to make a prediction for its arrival time there. This project will be exploring both static and dynamic models, but Countdown will be used as the baseline model for which to compare the models developed in this project against. \\

Only historical average models, regression models and machine learning models will be explored. The following parameters will also be investigated to see how relevant they are in predicting bus arrival times: weather, time of day and time of week, time of year (for example special holidays and school holidays), dwell time at bus stops and traffic conditions. \\

If this study is successful, the end product should be an application that predicts more accurately than the Countdown API, the arrival time of any bus on any route in London. Bus passengers want reliable and timely services, but if that is not possible, they want to know when a bus will actually turn up. Passengers want to avoid being left waiting at a bus stop for a bus that the board says is due, but has been saying it is due for the past three minutes. By providing a service that supplies users with reliable bus arrival times, it is intended that more people will consider the bus over the car as their primary mode of transport. However, it is entirely plausible that the dynamic models could prove less successful than the static model used by Countdown. If that is the case, a thorough evaluation will be done to investigate why. Add on a bit about how I will measure success here?

\clearpage

